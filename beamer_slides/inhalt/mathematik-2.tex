\begin{frame}[fragile]
	\frametitle{Mathematik -- Unterschiede bei der Darstellung}
	\begin{itemize}
		\item Formeln werden innerhalb der \keyword{math}-Umgebung anders dargestellt als in der \keyword{displaymath}-Umgebung
	\end{itemize}
	\begin{center}
		\begin{tabular}{c|c|c}
			\LaTeX-Befehl & \keyword{math} & \keyword{displaymath} \\ \hline
			\lstinline$\lim_{x \ra \infty} \ra 0$
			&
			\begin{math}
				\lim_{x \to \infty} \ra 0
			\end{math}
			&\begin{minipage}{4cm}
				\begin{displaymath}
					\lim_{x \ra \infty} \ra 0
				\end{displaymath}
			\end{minipage}
		\end{tabular}
	\end{center}
	\begin{itemize}
		\item mit dem Befehl \befehl{displaystyle} bzw. \befehl{textstyle} kann das jeweils andere Verhalten erzwungen werden
	\end{itemize}
	\latexBeispielDirekt{Beispiel: \keyword{textstyle} und \keyword{displaystyle}}{examples/Formeln_Darstellung/Formeln_Darstellung}
\end{frame}

\begin{frame}
	\frametitle{Mathematik -- Symbole}
	
	\emphword{Logik}
	\begin{center}
		\begin{tabular}{ll|ll|ll|ll|ll}
			\befehl{exists} & $\exists$ & \befehl{forall} & $\forall$ & \befehl{neg} & $\neq$ & \befehl{in} & $\in$ & \befehl{emptyset} & $\emptyset$ \\
			\befehl{notin} & $\notin$ & \befehl{ni} & $\ni$ & \befehl{land} & $\land$ & \befehl{lor} & $\lor$ & \befehl{varnothing} & $\varnothing$\\
			\befehl{setminus} & $\setminus$ & \befehl{implies} & $\implies$ & \befehl{iff} & $\iff$ & \befehl{to} & $\to$ & \befehl{top} & $\top$
		\end{tabular}
	\end{center}\vspace{1cm}
	
	\emphword{Pfeile} \vspace{-0.9cm}
	\begin{center}
		\begin{tabular}{ll|ll|ll|ll}
			\befehl{leftarrow} & $\la$ & \befehl{rightarrow} & $\ra$ & \befehl{uparrow} & $\uparrow$ & \befehl{downarrow} & $\downarrow$ \\
			\befehl{Leftarrow} & $\La$ & \befehl{Rightarrow} & $\Ra$ & \befehl{Uparrow} & $\Uparrow$ & \befehl{Downarrow} & $\Downarrow$ \\
			\befehl{leftrightarrow} & $\leftrightarrow$ & \befehl{Leftrightarrow} & $\Leftrightarrow$ & \befehl{nearrow} & $\nearrow$ & \befehl{searrow} & $\searrow$ \\
			 \befehl{updownarrow} & $\updownarrow$ & \befehl{Updownarrow} & $\Updownarrow$ & \befehl{swarrow} & $\swarrow$ & \befehl{nwarrow} & $\nwarrow$ \\
			 \befehl{longleftarrow} & $\longleftarrow$ & \befehl{Longleftarrow} & $\Longleftarrow$ & \befehl{mapsto} & $\mapsto$ & \befehl{leadsto} & $\leadsto$ \\
			 \befehl{longrightarrow} & $\longrightarrow$ & \befehl{Longrightarrow} & $\Longrightarrow$
		\end{tabular}
	\end{center}
\end{frame}

\begin{frame}
	\frametitle{Mathematik -- Symbole}
	\emphword{Sonstige}
	\begin{center}
		\begin{tabular}{ll|ll|ll|ll|ll}
			\befehl{partial} & $\partial$ & \befehl{nabla} & $\nabla$ & \befehl{infty} & $\infty$ & \befehl{ell} & $\ell$ & \befehl{imath} & $\imath$ \\
			\befehl{jmath} & $\jmath$ & \befehl{Re} & $\Re$ & \befehl{Im} & $\Im$ & \befehl{dots} & $\dots$ & \befehl{cdots} & $\cdots$ \\
			\befehl{vdots} & $\vdots$ & \befehl{ddots} & $\ddots$ & \befehl{ldots} & $\ldots$ & \befehl{pm} & $\pm$ & \befehl{mp} & $\mp$
		\end{tabular}
	\end{center} \vspace{1cm}
	

\end{frame}


\begin{frame}
	\frametitle{Mathematik -- Griechische Buchstaben}
	\vspace{-0.5cm}
	\begin{center}
\begin{tabular}{ll|ll||ll}
\befehl{alpha}      & $\alpha$  & 
\befehl{xi}         & $\xi$ &
\befehl{Gamma}      & $\Gamma$ \\

\befehl{beta}       & $\beta$  & 
\befehl{pi}         & $\pi$ &
\befehl{Delta}      & $\Delta$ \\

\befehl{gamma}      & $\gamma$  & 
\befehl{varpi}      & $\varpi$ &
\befehl{Theta}      & $\Theta$ \\

\befehl{delta}      & $\delta$  & 
\befehl{rho}        & $\rho$ &
\befehl{Lambda}     & $\lambda$ \\

\befehl{epsilon}    & $\epsilon$  & 
\befehl{varrho}     & $\varrho$ &
\befehl{Xi}         & $\Xi$ \\

\befehl{varepsilon} & $\varepsilon$  & 
\befehl{sigma}      & $\sigma$ &
\befehl{Pi}         & $\Pi$ \\

\befehl{zeta}       & $\zeta$  & 
\befehl{varsigma}   & $\varsigma$ &
\befehl{Sigma}      & $\Sigma$ \\

\befehl{eta}        & $\eta$  & 
\befehl{tau}        & $\tau$ &
\befehl{Upsilon}    & $\Upsilon$ \\

\befehl{theta}      & $\theta$  & 
\befehl{upsilon}    & $\upsilon$ &
\befehl{Phi}        & $\Phi$ \\

\befehl{vartheta}   & $\vartheta$  & 
\befehl{phi}        & $\phi$ &
\befehl{Psi}        & $\Psi$ \\

\befehl{iota}       & $\iota$  & 
\befehl{varphi}     & $\varphi$ &
\befehl{Omega}      & $\Omega$ \\

\befehl{kappa}      & $\kappa$  & 
\befehl{chi}        & $\chi$ \\

\befehl{lambda}     & $\lambda$  & 
\befehl{psi}        & $\psi$ &
                & \\

\befehl{mu}         & $\mu$  & 
\befehl{omega}      & $\omega$ &
                & \\

\befehl{nu}         & $\nu$ &
                &       &
                & \\

\end{tabular}
\end{center}
\end{frame}

\begin{frame}
	\frametitle{Mathematik -- binäre Operatoren}
	\begin{center}
\begin{tabular}{ll|ll|ll}
\befehl{leq} & $\leq$ &
\befehl{geq} & $\geq$ &
\befehl{equiv} & $\equiv$
\\
\befehl{models} & $\models$ &
\befehl{prec} & $\prec$ &
\befehl{succ} & $\succ$
\\
\befehl{sim} & $\sim$ &
\befehl{perp} & $\perp$ &
\befehl{preceq} & $\preceq$
\\
\befehl{succeq} & $\succeq$ &
\befehl{simeq} & $\simeq$ &
\befehl{mid} & $\mid$
\\
\befehl{ll} & $\ll$ &
\befehl{gg} & $\gg$ &
\befehl{asymp} & $\asymp$ 
\\
\befehl{parallel} & $\parallel$ &
\befehl{subset} & $\subset$ &
\befehl{supset} & $\supset$
\\
\befehl{approx} & $\approx$ &
\befehl{bowtie} & $\bowtie$ &
\befehl{subseteq} & $\subseteq$
\\
\befehl{supseteq} & $\supseteq$ &
\befehl{cong} & $\cong$ &
\befehl{sqsubset} & $\sqsubset$
\\
\befehl{sqsupset} & $\sqsupset$ &
\befehl{neq} & $\neq$ &
\befehl{smile} & $\smile$
\\
\befehl{sqsubseteq} & $\sqsubseteq$ &
\befehl{sqsupseteq} & $\sqsupseteq$ &
\befehl{doteq} & $\doteq$
\\
\befehl{frown} & $\frown$ &
\befehl{in} & $\in$ &
\befehl{ni} & $\ni$
\\
\befehl{propto} & $\propto$ &
= & $=$ &
\befehl{vdash} & $\vdash$
\\
\befehl{dashv} & $\dashv$ &
< & $<$ &
> & $>$
\end{tabular}
\end{center}
\end{frame}

\begin{frame}
	\frametitle{Mathematik -- Negation von Operatoren}
	\begin{itemize}
		\item oftmals kann ein Operator negiert werden, indem der Befehl \befehl{not} vorgestellt wird
		\begin{center}
			\begin{tabular}{ll|ll|ll}
				\befehl{not<} & $\not<$ & \befehl{not}\befehl{leq} & $\not\leq$ & \befehl{not}\befehl{subseteq} $\not\subseteq$
			\end{tabular}
		\end{center}
		\item einige Operatoren haben dafür jedoch spezielle Symbole
		\begin{center}
			\begin{tabular}{ll|ll}
				\befehl{not=} & $\not=$ & \befehl{neq} & $\neq$ \\
				\befehl{not}\befehl{in} & $\not\in$ & \befehl{notin} & $\notin$
			\end{tabular}
		\end{center}
	\end{itemize}
\end{frame}

\begin{frame}
	\frametitle{Mathematik  -- gestapelte Operatoren}
	\begin{itemize}
		\item manchmal werden Anmerkungen an Relationen geschrieben oder eigene Operatoren definiert
		\item dazu eignet sich der \befehl{stackrel} Befehl
	\end{itemize}
	\vfill
	\latexBeispielDirekt{Beispiel: \befehl{stackrel} Befehl}{examples/Mathematik_stackrel/Mathematik_stackrel}
\end{frame}

\begin{frame}
	\frametitle{Mathematik -- Funktionen}
	\begin{center}
\begin{tabular}{lllll}
\befehl{arccos} \quad &
\befehl{arcsin} \quad &
\befehl{arctan} \quad &
\befehl{arg} \quad &
\befehl{cos} \\[0.5cm]
\befehl{cosh} &
\befehl{cot} &
\befehl{coth} &
\befehl{csc} &
\befehl{deg} \\[0.5cm]
\befehl{det} &
\befehl{dim} &
\befehl{exp} &
\befehl{gcd} &
\befehl{hom} \\[0.5cm]
\befehl{inf} &
\befehl{ker} &
\befehl{lg} &
\befehl{lim} &
\befehl{liminf} \\[0.5cm]
\befehl{limsup} &
\befehl{ln} &
\befehl{log} &
\befehl{max} &
\befehl{min} \\[0.5cm]
\befehl{Pr} &
\befehl{sec} &
\befehl{sin} &
\befehl{sinh} &
\befehl{sup} \\[0.5cm]
\befehl{tan} &
\befehl{tanh} &
\end{tabular}
\end{center}
\end{frame}

\begin{frame}
	\frametitle{Mathematik -- Hoch- und Tiefstellung}
	\begin{itemize}
		\item um Exponenten oder Indizes anzugeben, benutzt man zur Hochstellung das Zeichen \lstinline$^$ und zur Tiefstellung das Zeichen \lstinline$_$
		\item möchte man mehr als ein Zeichen hoch- oder tiefstellen, muss man den Term in geschweiften Klammern angeben
		\item die Reihenfolge ist egal
		\item mit den Befehlen \befehl{limits} und \befehl{nolimits} kann man die Darstellung für das Hoch- bzw. Tiefstellen beinflussen
	\end{itemize}
	\vfill
	\latexBeispielDirekt{Beispiel: Hoch- und Tiefstellung}{examples/Mathematik_HochTief/Mathematik_HochTief}
\end{frame}

\begin{frame}[fragile]
	\frametitle{Mathematik -- Brüche}
	\begin{itemize}
		\item der \befehl{frac} Befehl dient der Darstellung von Brüchen und erfordert zwei Argumente
	\end{itemize}
	\vfill
	\latexBeispielDirekt{Beispiel: Brüche}{examples/Mathematik_frac/Mathematik_frac}
\end{frame}

\begin{frame}[fragile]
	\frametitle{Mathematik -- Wurzel und Binomialkoeffizient}
	\emphword{Wurzel}
	\begin{itemize}
		\item der Wurzelbefehl heißt \befehl{sqrt}
		\item optional kann der Parameter \keyword{n} angeben werden ($n = 2$:  Quadratwurzel)
	\end{itemize}
	\vfill
	\emphword{Binomialkoeffizient}
	\begin{itemize}
		\item zur Darstellung des Binomialkoeffizienten dient der Befehl \befehl{choose}
		\item das erste Argument wird oben, das zweite unten gesetzt
	\end{itemize}
	\vfill
	\latexBeispielDirekt{Beispiel: Wurzeln, Binomialkoeffizienten}{examples/Mathematik_Wurzel_Binom/Mathematik_Wurzel_Binom}
\end{frame}

\begin{frame}[fragile]
	\frametitle{Mathematik -- Klammerung} \vspace{-0.6cm}
	\begin{itemize}
		\item Klammern: \keyword{() [] \{\} ||} \befehl{langle}\befehl{rangle} \befehl{lfloor}\befehl{rfloor} \befehl{lceil}\befehl{rceil}
			\[ (\,) \quad [\,] \quad \{\,\} \quad |\,| \quad \langle \, \rangle \quad \lfloor \, \rfloor \quad \, \lceil \, \rceil \]
		\item die Größe der Klammern kann manuell durch die Befehle \befehl{big}, \befehl{Big}, \befehl{bigg}, \befehl{Bigg} variert werden
			\[ ( \big( \Big( \bigg( \Bigg( \, \Bigg) \bigg) \Big) \big) ) \]
		\item besser: die Klammergröße kann automatisch gesetzt werden
		\item dazu setzt man den einzuklammernden Begriff zwischen die Befehle \befehl{left} und \befehl{right}; direkt darauf folgt das zu verwendende Klammerzeichen\\
		\item möchte man auf einer Seite keine Klammern haben, verwendet man als Klammerzeichen \keyword{.} (Punkt)
		\item Beispiel: \lstinline$\left\{ 1 - \left| \frac{1}{2} \right| \dots \right.$
		\[
			\left\{ 1 - \left| \frac{1}{2} \right| \dots \right.
		\]
	\end{itemize}
\end{frame}

\begin{frame}[fragile]
	\frametitle{Mathematik -- Matrizen}
	\begin{itemize}
		\item die \keyword{array}-Umgebung entspricht der \keyword{tabular}-Umgebung im Mathematik-Modus
		\item die Inhalte der Zellen sind automatisch im Mathematik-Modus gesetzt
		\item bekannte Befehle für Tabellen (z. B. \befehl{multicolumn}, \befehl{hline}, \dots) können verwendet werden
	\end{itemize}
	\vfill
	\latexBeispielDirekt{Beispiel: Matrizen}{examples/Mathematik_Matrix/Mathematik_Matrix}
\end{frame}

\begin{frame}[fragile]
	\frametitle{Mathematik -- Fallunterscheidungen}
	\begin{itemize}
		\item für Fallunterscheidungen steht die \keyword{cases}-Umgebung zur Verfügung
		\item kann auch mittels Klammerung und der \keyword{array}-Umgebung selbst erstellt bzw. individualisiert werden
	\end{itemize}
	\vfill
	\latexBeispielDirekt{Beispiel: Fallunterscheidungen}{examples/Mathematik_cases/Mathematik_cases}
\end{frame}

\begin{frame}[fragile]
	\frametitle{Mathematik -- Schriftgröße}
	\begin{itemize}
		\item die Schriftgröße kann im Mathematik-Modus durch folgende Befehl geändert werden:
		\begin{center}
			\begin{tabular}{ll}
				\befehl{displaystyle} & $\displaystyle \frac{1}{2}$ \\[0.5cm]
				\befehl{textstyle} & $\textstyle \frac{1}{2}$ \\[0.5cm]
				\befehl{scriptstyle} & $\scriptstyle \frac{1}{2}$
			\end{tabular}
		\end{center}
	\end{itemize}
\end{frame}

\begin{frame}[fragile]
	\frametitle{Mathematik -- Akzente}
	\begin{center}
		\begin{tabular}{ll|ll|ll}
			\verb|a'| & $a'$ &
			\verb|a''| & $a''$ &
			\verb|a'''| & $a'''$ \\
			\verb|\bar{a}| & $\bar{a}$ &
			\verb|\overline{a}| & $\overline{a}$ &
			\verb|\underline{a}| & $\underline{a}$ \\
			\verb|\hat{a}| & $\hat{a}$ &
			\verb|\widehat{ab}| & $\widehat{ab}$ &
			\verb|\check{a}| & $\check{a}$ \\
			\verb|\tilde{a}| & $\tilde{a}$ &
			\verb|\widetilde{ab}| & $\widetilde{ab}$ &
			\verb|\vec{a}| & $\vec{a}$ \\
			\verb|\dot{a}| & $\dot{a}$ &
			\verb|\ddot{a}| & $\ddot{a}$
		\end{tabular}
	\end{center}
\end{frame}

\begin{frame}[fragile]
	\frametitle{Mathematik -- horizontale Abstände}
	\begin{center}
		\begin{tabular}{l|c|l}
		Befehl & Breite & Beschreibung\\ \hline
			\verb|\qquad| & $\overline{\qquad}$ & $2\times$ quad \\
			\verb|\quad| & $\overline{\quad}$ & so breit wie ein Zeichen hoch ist \\
			\textvisiblespace~ (Leerzeichen)          & $\overline{~}$ & Zeichenabstand \\
			\verb|\,| & $\overline{\,}$ & $\frac{3}{18} \times$ quad \\
			\verb|\:| & $\overline{\:}$ & $\frac{4}{18} \times$ quad \\
			\verb|\;| & $\overline{\;}$ & $\frac{4}{18} \times$ quad \\
			\verb|\!| & $\overline{\,}$ & $-\frac{3}{18} \times$ quad \\
		\end{tabular}
	\end{center}
	\vfill
	\latexBeispielDirekt{Beispiel: horizontaler Abstand}{examples/Mathematik_H_Abstand/Mathematik_H_Abstand}
\end{frame}
