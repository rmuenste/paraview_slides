\begin{frame}[fragile]
  \frametitle{Eigene \LaTeX-Befehle}
        \begin{itemize}
          \item eigene Befehle können mittels \befehl{newcommand} angelegt werden
          \item allgemeine Form:
            \begin{itemize}
              \item \befehl{newcommand\{befehlsname\}\{definition\}}
              \item \befehl{newcommand\{befehlsname\}[n]\{definition\}}
              \item \befehl{newcommand\{befehlsname\}[n][default]\{definition\}}
            \end{itemize}
                \begin{center}
                  \begin{tabular}{rl}
                    \emphkeyword{befehlsname} & Name des Befehls (muss mit \textbackslash{} beginnen)\\
                                \emphkeyword{n} & Anzahl der Parameter \\
                                \emphkeyword{default} & Vorgabewert für optionale Parameter\\
                                \emphkeyword{definition} & alles was beim Aufruf ausgeführt werden soll
                  \end{tabular}
                \end{center}
              \item mit \befehl{renewcommand} kann ein bereits vorhandener Befehl ersetzt werden
        \end{itemize}
\end{frame}

\begin{frame}[fragile]
  \frametitle{Eigene \LaTeX-Befehle -- Beispiel}

        \latexBeispielDirekt{Eigene \LaTeX-Befehle -- Beispiel}{examples/Eigene_Befehle/Eigene_Befehle}
\end{frame}

\begin{frame}[fragile]
  \frametitle{Eigene \LaTeX-Umgebungen}
        \begin{itemize}
          \item eigene Umgebungen können mittels \befehl{newenvironment} angelegt werden
          \item allgemeine Form:
            \begin{itemize}
              \item \befehl{newenvironment\{umgebung\}\{vorher\}\{nachher\}}
              \item \befehl{newenvironment\{umgebung\}[n]\{vorher\}\{nachher\}}
              \item \befehl{newenvironment\{umgebung\}[n][default]\{vorher\}\{nachher\}}
            \end{itemize}
                \begin{center}
                  \begin{tabular}{rl}
                    \emphkeyword{umgebung} & Name der Umgebung (ohne \textbackslash{}) \\
                                \emphkeyword{n} & Anzahl der Parameter \\
                                \emphkeyword{vorage} & Vorgabewert für optionale Parameter\\
                                \emphkeyword{vorher} & Befehle, die vor Beginn der Umgebung ausgeführt werden\\
                                \emphkeyword{nachher} & Befehle, die nach Ende der Umgebung ausgeführt werden
                  \end{tabular}
                \end{center}
              \item mit \befehl{renewenvironment} kann eine bereits vorhandene Umgebung ersetzt werden
        \end{itemize}
\end{frame}

\begin{frame}[fragile]
  \frametitle{Eigene \LaTeX-Umgebungen -- Beispiel}

        \latexBeispielDirekt{Eigene \LaTeX-Umgebungen -- Beispiel}{examples/Eigene_Umgebung/Eigene_Umgebung}
\end{frame}

\begin{frame}[fragile]
  \frametitle{Eigene Zähler}
  \vspace{-1cm}
        \begin{center}
          \begin{tabular}{lll}
            \befehl{newcounter\{zaehler\}} & \multicolumn{2}{l}{neuen Zähler namens \keyword{zaehler} anlegen}\\
                        \befehl{newcounter\{zaehler\}[depend]} & \multicolumn{2}{l}{neuen Zähler namens \keyword{zaehler} anlegen, der bei}\\& \multicolumn{2}{l}{Veränderung von \keyword{depend} zurückgesetzt wird}\\
                        \befehl{stepcounter\{zaehler\}} & \multicolumn{2}{l}{Zähler \keyword{zaehler} um eines erhöhen}\\
                        \befehl{setcounter\{zaehler\}\{wert\}} & \multicolumn{2}{l}{Zähler \keyword{zaehler} auf Wert \keyword{wert} setzen}\\
                        \befehl{addtocounter\{zaehler\}\{wert\}} & \multicolumn{2}{l}{Zähler \keyword{zaehler} um \keyword{wert} erhöhen} \\
                        \befehl{value\{zaehler\}} & \multicolumn{2}{l}{den Wert von \keyword{zaehler} auslesen}\\
                        \befehl{arabic\{zaehler\}} & Ziffer & 1, 2, 3, 4, \dots \\
                        \befehl{roman\{zaehler\}} & Römische Zahl (klein) & i, ii, iii, iv, \dots \\
                        \befehl{Roman\{zaehler\}} & Römische Zahl (groß) \hspace{1cm} & I, II, III, IV, \dots \\
                        \befehl{alph\{zaehler\}} & Kleinbuchstaben & a, b, c, d, \dots \\
                        \befehl{Alph\{zaehler\}} & Großbuchstaben & A, B, C, D, \dots
          \end{tabular}
        \end{center}
\end{frame}

