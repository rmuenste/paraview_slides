\begin{frame}[fragile]
	\frametitle{Grafiken}
	\begin{itemize}
		\item Grafiken können mit dem Befehl \befehl{includegraphics} eingebunden werden
		\item dieser benötigt das Paket \keyword{graphicx}, dass in der Präambel geladen werden muss:\\
		\lstinline$\usepackage{graphicx}$
		\item benutzt man \keyword{pdflatex} zum Kompilieren des Dokuments, so müssen die verwendeten Grafiken in PDF- oder Bitmap-Grafiken umgewandelt werden
		\item alternativ kann man auch das Paket \keyword{epstopdf} verwendet, dann muss \keyword{pdflatex} aber so gestartet werden:\\
		\kommandozeile{pdflatex --shell-escape}
	\end{itemize}
\end{frame}

\begin{frame}[fragile]
	\frametitle{Grafiken -- Beispiel} \vspace{-0.5cm}
	\latexBeispielDirekt{Beispiel: \befehl{includegraphics}}{examples/Grafiken_Beispiel/Grafiken_Beispiel} \par \vspace{1cm}
	\emphword{Optionale Parameter}
	\begin{center}
		\begin{tabular}{ll}
			\keyword{width=b} & Breite vorgeben  (z. B. in \keyword{cm})\\
			\keyword{height=h} & Höhe vorgeben (z. B. in \keyword{cm})\\
			\keyword{keepaspectratio=k} & Seitenverhältnis beibehalten (\keyword{true} oder \keyword{false}) \\
			\keyword{scale=s} & skalieren  (Skalierungsfaktor angeben)\\
			\keyword{angle=a} & rotieren (Winkel in Grad angeben) \\
			\keyword{trim=l b r t} & Bild zuschneiden
		\end{tabular}
	\end{center}
\end{frame}

\begin{frame}[fragile]
	\frametitle{Grafiken -- \keyword{picture}-Umgebung}
	\latexBeispielDirektKlein{Beispiel: \keyword{picture}-Umgebung}{examples/Grafiken_pictures/Grafiken_pictures}
\end{frame}
